\chapter{アクターモデルの形式化}
\label{chapter:formalization}

本章では,Actarioにおけるアクターモデルの形式化手法について説明する.
まず,意味論の形式化,そして名前付けの方法,その名前付けの方法が一意な名前を生成するということの証明の説明を行う.


\section{意味論の形式化}

アクターモデルの操作的意味を配置のラベル付き遷移システムとして定式化する.
これ以降用いる記号を図\ref{expr:formalization:config}と定義する.
以下はその説明である.
%% また,Actarioではこれは図\ref{code:formalization:config}と定義している.

\begin{description}[style=nextline,leftmargin=12pt,parsep=0pt]
\item[\textit{Configuration}]
  アクターシステムの配置を表す.実体はアクターの集合である.
\item[\textit{Actor}]
  アクターを表す.アクターの名前,まだ実行していないアクションの列,生成番号,振る舞いのテンプレート,メッセージキューから成る.
\item[\textit{Name}]
  アクターの名前を表す.名前は,トップレベルのアクターか,生成されたアクターのどちらかであるかを表す.トップレベルのアクターとはつまり親アクターがいないアクターであり,システムの名称を表す文字列を引数にとる.また,トップレベルのアクターをトップレベルアクターと呼ぶ.生成されたアクターは,親アクターの名前と,その親アクターが何番目に生成したアクターかを表す自然数を引数にとる.
\item[\textit{Message}]
  メッセージは,アクターの名前,基本的なデータ型(自然数,文字列),およびその組み合わせ(タプル)から構成される.
\item[\textit{Behavior State}]
  アクターの振る舞いを表す.メッセージを受け取り,アクションの列を返す関数である.
\item[\textit{Actions State}]
  アクターのアクションの列を表す.アクションには,他のアクターにメッセージを送る\textsf{send},新しくアクターを生成する\textsf{new},自分自身の名前を得る\textsf{self},新しい状態になり次のメッセージの待ち状態になる\textsf{become}の4つがある.
  \textsf{become}以外の各アクションは,最後の引数にこのアクションの継続を表すものをとる.つまり,メッセージの処理は必ず\textsf{become}で終わることになる.
  また,ここでの\textit{State}は型引数であり,\textsf{become}の引数はこの型の値でなければならない.
\item[\textit{Label}]
  遷移システムのラベルを表す.\textsf{Receive},\textsf{Send},\textsf{New},\textsf{Self}はそれぞれ消費したアクションが\textsf{become},\textsf{send},\textsf{new},\textsf{self}であったときのラベルである.
\end{description}


\begin{figure}[t]
  \begin{displaymath}
    \begin{array}{rclcl}
      c & \in & \textit{Configuration} & =   & \textit{Set Actor} \\
      a & \in & \textit{Actor}  & =   & \textit{Name} \times \textit{Actions State} \times \mathbb{N} \times \\
        &     &                 &     & (\textit{State} \rightarrow \textit{Behavior State}) \times \textit{List Message} \\
      n & \in & \textit{Name}   & ::= & \textsf{toplevel}(str) \\
        &     &                 &   | & \textsf{generated}(g, n) \\
      m & \in & \textit{Message} & =  & \textit{Name} + \textit{PrimVal} + \\
        &     &                 &     & \textit{Message} \times \cdots \times \textit{Message} \\
      b & \in & \textit{Behavior State} & = & \textit{Message} \rightarrow \textit{Actions State} \\
      \alpha & \in & \textit{Actions State} & ::= & \textsf{send}(n, m, \alpha) \\
        &     &                 &   | & \textsf{new}(t, s, \kappa) \\
        &     &                 &   | & \textsf{self}(\kappa) \\
        &     &                 &   | & \textsf{become}(s) \\
      l & \in & \textit{Label}  & ::= & \textsf{Receive}(n, m) \\
        &     &                 &   | & \textsf{Send}(n, n, m) \\
        &     &                 &   | & \textsf{New}(n) \\
        &     &                 &   | & \textsf{Self}(n) \\
      \kappa & \in & \textit{Name} \rightarrow \textit{Actions State} \\
      s & \in & \textit{State} & & \\
      g & \in & \mathbb{N} & & \\
      t & \in & \textit{State} \rightarrow \textit{Behavior State} \\
      q & \in & \textit{List Message}
      str & \in & \textit{String}
    \end{array}
  \end{displaymath}
  \caption{記号の定義}\label{expr:formalization:config}
\end{figure}


\begin{figure}
  \lstinputlisting{./code/formalization/config.v}
  \caption{Actarioでの記号の定義}\label{code:formalization:config}
\end{figure}

以上の記号を用いてアクターモデルのラベル付き遷移システムは図\ref{expr:formalization:semantics}のように定式化できる.
$\overset{l}{\leadsto}$は二項関係であり,左手の配置がラベル$l$によって右手の配置に遷移するということを表している.
遷移関係は4つあり,それぞれ\textsf{Receive}ラベル,\textsf{Send}ラベル,\textsf{New}ラベル,\textsf{Self}ラベルによる遷移である.それぞれについての説明を以下に記す.

\begin{description}[style=nextline,leftmargin=12pt,parsep=0pt]
\item[$\textsf{Receive}(n_{\textrm{to}}, m)$]
  $n_{\textrm{to}}$が自身のメッセージキューの先頭にある$m$というメッセージを受け取り,そのメッセージによってアクションの列を生成する遷移である.
\item[$\textsf{Send}(n_{\textrm{from}}, n_{\textrm{to}}, m)$]
  $n_{\textrm{from}}$から$n_{\textrm{to}}$に向けてメッセージ$m$を送信するという遷移である.
  送信されたメッセージは即座に$n_{\textrm{to}}$のメッセージキューに入れられる.
\item[$\textsf{New}(n')$]
  $n'$という名前を持つアクター(ここでは子アクターと呼ぶ)を生成する遷移である.
  生成する側のアクター(ここでは親アクターと呼ぶ)の名前と生成番号によって名前がつけられる.
  親アクターは子アクターを生成した後,自身の生成番号を1増やす.
  子アクターは生成番号が0,メッセージ待ち状態でメッセージキューは空の状態で生成される.
\item[$\textsf{Self}(n)$]
  $n$というアクターが自分自身の名前を得る遷移である.
\end{description}


\begin{figure}[t]\centering
  \begin{displaymath}
    \begin{array}{rcl}
      c \cup \{(n_{\textrm{to}}, \textsf{become}(s), g, t, q)\} &
      \overset{\textsf{Receive}(n_{\textrm{to}}, m)}{\leadsto} &
      c \cup \{(n_{\textrm{to}}, t(s)(m), g, t, q)\}
      \hfill \textsc{(Receive)} \\[2ex]

      c \cup \{(n_{\textrm{from}}, \textsf{send}(n_{\textrm{to}}, m, \alpha), g, t, q),
      & &
      c \cup \{(n_{\textrm{from}}, \alpha, g, t, q), \\
      (n_{\textrm{to}}, \alpha', g', t', q')\} &
      \overset{\textsf{Send}(n_{\textrm{from}}, n_{\textrm{to}}, m)}{\leadsto} &
      \hspace{5ex} (n_{\textrm{to}}, \alpha', g', t', q' \doubleplus [m])\}
      \hfill \textsc{(Send)} \\[2ex]

      & & c \cup \{(n, \kappa(n'), g + 1, t, q), \\
      c \cup \{(n, \textsf{new}(t', s, \kappa), g, t, q)\} &
      \overset{\textsf{New}(n')}{\leadsto} &
      \hspace{5ex} (n', \textsf{become}(s), 0, t', [])\} \\
      & & \textrm{where}\ n' := \textsf{generated}(g, n)
      \hfill \textsc{(New)} \\[2ex]

      c \cup \{(n, \textsf{self}(\kappa), g, t, q)\} &
      \overset{\textsf{Self}(n)}{\leadsto} &
      c \cup \{(n, \kappa(n), g, t, q)\}
      \hfill \textsc{(Self)}
    \end{array}
  \end{displaymath}
  \caption{ラベル付き遷移システム}\label{expr:formalization:semantics}
\end{figure}

\section{名前付け}

アクターモデルではメッセージ送信の際に宛先のアクターの名前を指定してメッセージを送るため,アクターの名前が一意になるという性質は重要である.
一般的なアクターの実装では,一意な名前付けと名前の管理をするために,副作用を使った名前付けを行うことが多い.例えば,グローバル変数に自然数を持っておき,名前を生成する度にインクリメントする方法や,メモリ上に確保されるアクターのアドレスをアクターの名前にしてしまう方法がある.
しかし,Coqのような副作用を記述できない純粋な言語では,一意な名前付けは問題になりやすい.
Actarioでは,アクターの名前を,親の名前とその親が何番目に生成したアクターかを表す自然数のペアで表現することで,一意な名前付けを実現している.

本節ではこの操作的意味論の名前付けの方法によって生成された名前は必ず一意になるということを証明する.
また,これ以降,アクターが内部状態として持つ次に生成する番号を\emph{生成番号},アクターの名前に含まれる親アクターが何番目に生成したアクターかという番号を\emph{世代番号}と呼ぶ.

\subsection{名前の一意性の証明}

生成された名前の一意性の証明に入る.
これを証明するために,\emph{\transinv(trans invariant)}という遷移の間で変わらない名前に関する性質を定義する.
\transinv は配置に関する述語であり,以下のように3つの述語,\emph{\chain (chain)}, \emph{\fresh (gen fresh)}, \emph{\nodup (no\_dup)}の組で定義される.

\begin{definition}{\transinv}
\begin{displaymath}
  \begin{array}{l}
    \texttt{trans\_invariant}(c) :=
    \texttt{chain}(c) \wedge \texttt{gen\_fresh}(c) \wedge \texttt{no\_dup}(c)
  \end{array}
\end{displaymath}
\end{definition}

\texttt{chain}, \texttt{gen\_fresh}, \texttt{no\_dup}の簡単な説明は以下のとおりである.

\begin{description}[style=nextline,leftmargin=12pt,parsep=0pt]
\item[\chain]
  配置内の各アクターについて,そのアクターが他のアクターから生成されたものであるなら親アクターはその配置に存在している.
\item[\fresh]
  配置内の各アクターについて,そのアクターが次に生成するアクターの名前は配置内で新しい名前である.
\item[\nodup]
  配置内のすべてのアクターの名前は一意である.
\end{description}

証明の説明に入る前に,この節で用いる関数をいくつか定義する.


\begin{description}[style=nextline,leftmargin=12pt,parsep=0pt]
\item[\texttt{parent} $: \textit{Actor} \rightarrow \textit{Actor}$]
  アクターを受け取り,そのアクターの親アクターを返す.親がいない場合は,\texttt{nothing} を返す.
\item[\texttt{gen\_number} $: \textit{Actor} \rightarrow \mathbb{N}$]
  アクターを受け取り,そのアクターの名前の世代番号を返す.
  トップレベルアクターの場合は,\texttt{nothing} を返す.
\item[\texttt{next\_number} $: \textit{Actor} \rightarrow \mathbb{N}$]
  アクターを受け取り,そのアクターの生成番号を返す.
\item[\texttt{name} $: \textit{Actor} \rightarrow \textit{Name}$]
  アクターを受け取り,そのアクターの名前を返す.
\item[\texttt{names} $: \textit{Set(Actor)} \rightarrow \textit{Set(Name)}$]
  アクターの集合を受け取り,その集合の各アクターの名前の集合を返す.
\end{description}

\subsubsection{\chain}

まず,\transinv の一つ目の述語である\emph{\chain (chain)}という,配置に関する述語を定義する.
\chain とは,配置の中に含まれるアクターについて,そのアクターが他のアクターによって生成されたものであるならば,そのアクターも配置の中に含まれる,という述語である.
つまり,トップレベルアクターを起点とする木になるということである.
ここで,トップレベルアクターが複数ある場合でも,トップレベルアクター以外で親アクターが存在しないアクターが存在しなければ\chain は成り立つということに注意されたい.
\chain は以下のように定義される.

\begin{definition}{\chain}
\begin{displaymath}
  \begin{array}{l}
    \texttt{chain}(c) :=
    \forall a \in c, \forall p, p = \texttt{parent}(a) \Rightarrow p \in c
  \end{array}
\end{displaymath}
\end{definition}

ここで,任意の遷移について\chain は保存される\chainpreserv を証明する.
証明は遷移のラベルによる場合分けで行う.
\chain はアクターの名前によってのみで決定され,かつ名前が変更されうる遷移ラベルは\textsc{New}しかないので,\textsc{New}のみ考えればよい.

\begin{lemma}{\chainpreserv}
\begin{displaymath}
  \begin{array}{l}
    \forall c, c' \in \textit{Configuration}, \forall l \in \textit{Label},
    \texttt{chain}(c) \wedge c \overset{l}{\leadsto} c' \Rightarrow \texttt{chain}(c')
  \end{array}
\end{displaymath}
\end{lemma}

%% \begin{figure}[tp]
%%   \lstinputlisting{./code/formalization/chain_preservation.v}
%%   \label{code:formalization:chain-preservation}
%%   \caption{連鎖保存性の証明}
%% \end{figure}

\subsubsection{\fresh}

次に,\fresh を定義する.
\fresh は任意のアクターが次に生成する名前は必ず一意になるという述語であるが,
Actarioではこれを以下のように定義することで実現している.
%% We define \texttt{gen\_fresh} predicate that, for each actor in the configuration, the name of its child is always fresh.
%% The definition of \texttt{gen\_fresh} is complicated a little.
%% We translate the proposition that next generated name is fresh to the following.

\begin{definition}{\fresh}
\begin{displaymath}
  \begin{array}{l}
    \texttt{gen\_fresh}(c) := \\
    \quad \forall a \in \texttt{actors}(c), \forall p \in \texttt{actors}(c), p = \texttt{parent}(a) \Rightarrow \\
    \quad \quad \quad \texttt{gen\_number}(a) < \texttt{next\_number}(p)
  \end{array}
\end{displaymath}
\end{definition}

アクターの名前は生成するアクターの名前とその生成番号のペアで生成されるため,この\fresh が成り立っていれば必ず一意な名前を生成するということは,あるアクターの名前とそのアクターの生成番号からなる名前を持つようなアクターが存在しないことから成り立つ.
ただし,次に生成するアクターの名前は一意でもその次に生成するアクターは一意ではないこともある.
例えば,同じ名前でかつ次の生成番号も同じという2つのアクターがいた場合,まず片方のアクターが生成するアクターの名前は一意だが,その次にもう片方のアクターがアクターを生成したとすると,名前が被ってしまう(これを第一パターンとする).
また,親アクターがシステム内に存在せずに,親の親は存在しているという場合,親の親が次に生成するアクターの名前は被らないが,その子アクターが次に生成する名前は他のアクターの名前と被ってしまう可能性がある(これを第二パターンとする).
このように,遷移前で\fresh が成り立っていても,いくつか遷移したあとで一意ではない名前を生成することはありうる.
以上から,遷移後も\fresh が成り立つようにするためには,遷移前は\fresh という条件だけでは足りない.
第一パターンは\nodup を,第二パターンは\chain を前提に含めることで防ぐことができるので,
\freshpreserv は以下のような補題となる.

\begin{lemma}{\freshpreserv}
\begin{displaymath}
  \begin{array}{l}
    \forall c, c' \in \textit{Configuration}, \forall l \in \textit{Label}, \\
    \quad \texttt{chain}(c) \wedge \texttt{gen\_fresh}(c) \wedge \texttt{no\_dup}(c) \wedge c \overset{l}{\leadsto} c' \Rightarrow \\
    \quad \texttt{gen\_fresh}(c')
  \end{array}
\end{displaymath}
\end{lemma}

\subsubsection{\nodup}

最後に\nodup を定義する.
一意性は,与えられた配置のすべてのアクターの名前が一意であるという述語である.
これは最終的に証明したい名前の一意性の定義と同じである.
一意性は以下のように定義できる.

\begin{definition}{\nodup}
\begin{displaymath}
  \begin{array}{l}
    \texttt{no\_dup}(c) :=
    \forall a \in c, \texttt{name}(a) \notin
    \texttt{names}(c \setminus \{a\})
  \end{array}
\end{displaymath}
\end{definition}

\chainpreserv,\freshpreserv と同様に,\noduppreserv も定義する.
遷移後も\nodup が成り立つためには遷移前の配置が\nodup と\fresh を満たす必要がある.

\begin{lemma}{\noduppreserv}
\begin{displaymath}
  \begin{array}{l}
    \forall c, c' \in \textit{Configuration}, \forall l \in \textit{Label}, \\
    \quad \texttt{gen\_fresh}(c) \wedge \texttt{no\_dup}(c) \wedge c \overset{l}{\leadsto} c' \Rightarrow \texttt{no\_dup}(c')
  \end{array}
\end{displaymath}
\end{lemma}

\subsubsection{名前の一意性の証明}

ここまでで\transinv の定義ができたので,この意味論によって動的に生成される名前が一意になるということの証明を行う.
まず,遷移の間で\transinv が保存されるという\emph{\transinvpreserv (trans invariant preservation)}を証明する.
\transinv は\chain,\fresh,\nodup から成るので,
\chainpreserv,\freshpreserv,\noduppreserv から,\transinvpreserv は明らかに成り立つ.

\begin{lemma}{\transinvpreserv}
  \begin{displaymath}
    \begin{array}{l}
      \forall c, c' \in \textit{Configuration}, \forall l \in \textit{Label}, \\
      \quad \texttt{trans\_invariant}(c) \wedge c \overset{l}{\leadsto} c' \Rightarrow
      \texttt{trans\_invariant}(c')
    \end{array}
  \end{displaymath}
\end{lemma}

次に初期状態について\transinv が成り立っていれば,任意回の遷移後も\transinv が成り立つという補題($\transinvpreserv^{\star}$)を証明する.

\begin{lemma}{$\transinvpreserv^{\star}$}
  \begin{displaymath}
    \begin{array}{l}
      \forall c, c' \in \textit{Configuration}, \forall l \in \textit{Label}, \\
      \quad \texttt{trans\_invariant}(c) \wedge c \overset{l}{\leadsto\star} c' \Rightarrow
      \texttt{trans\_invariant}(c')
    \end{array}
  \end{displaymath}
\end{lemma}

ここで,$c \overset{l}{\leadsto\star} c'$ は遷移の反射推移閉包である.
この証明は遷移についての帰納法と\transinvpreserv によって証明できる.

最後に,初期状態について\transinv が成り立っていれば名前の一意性は成り立つということを証明する.
\transinv は\nodup から成るので,$\transinvpreserv^{\star}$から明らかである.
\begin{theorem}{名前の一意性}
  \begin{displaymath}
    \begin{array}{l}
      \forall c, c' \in \textit{Configuration}, \forall l \in \textit{Label}, \\
      \quad \texttt{trans\_invariant}(c) \wedge c \overset{l}{\leadsto\star} c' \Rightarrow \texttt{no\_dup}(c')
    \end{array}
  \end{displaymath}
\end{theorem}
