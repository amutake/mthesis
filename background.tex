\chapter{背景知識}

\section{アクターモデル}

アクターモデルとは、非同期メッセージパッシングに基づいた並行計算のモデルである。
アクターと呼ばれる並行計算の主体が互いにメッセージを送り合うことで計算を進める。
対象となるアクターの名前を指定することでメッセージを送信する。


\subsection{振る舞い}

振る舞い(behavior)とは、アクターがメッセージを受け取って何をするか(action)というものである。
actionには3つのものがある。

\begin{itemize}
\item 新しいアクターを生成する
\item 他のアクターにメッセージ
\item 自分自身の振る舞いを変える
\end{itemize}


\subsection{配置}

配置(configuration)とは、システムの一瞬を切り取ったスナップショットのことである。
その時点でのアクターおよびまだ受け取られていないメッセージの集合から構成される。

\subsection{アクターの名前}

アクターのアドレスのことをアクターの名前という。
アクターの名前は一意である。


\section{定理証明支援系 Coq}

Coqとは、フランス国立情報学自動制御研究所(INRIA)で開発が進められている定理証明支援系と呼ばれるシステムの一つである。
定理証明支援系とは人間が形式的な形で証明を書きやすいように支援するシステムのことである。
定理証明支援系Coqでは以下のことが行える。

\begin{itemize}
\item プログラムの仕様の検証や数学的な命題の証明
\item 型理論に基づいた証明の機械的なチェック
\item タクティクによる柔軟な証明記述
\item プラグインシステムによる拡張
\item 他の言語へのコード抽出
\end{itemize}

\subsection{コード抽出}

CoqにはOCaml, Haskell, Schemeへのコード抽出器がデフォルトで内蔵されている。
抽出器はプラグインの形で実装されており、

\section{Erlang}

Erlang\cite{erlang}は高可用なシステムを作るために作られたプログラミング言語である。
並行計算のモデルとしてアクターモデルを採用している。Erlangにおいて、並行計算の主体はプロセスと呼ばれる。

\subsection{基本的な構文}

本論文で扱うErlangの基本的な構文について説明する。

\begin{description}
\item[変数]\mbox{}\\
  大文字から始まる識別子は変数である。Erlangでは変数は不変であり、再代入することはできない。
\item[アトム]\mbox{}\\
  小文字から始まる識別子はアトム(atom)と呼ばれるシンボルである。
\item[組]\mbox{}\\
  波括弧で囲まれ、カンマで句切られた項は組(tuple)である。例えば\lstinline|{Var, atom, 42}|は変数\lstinline{Var}、アトム\lstinline{atom}、数値\lstinline{42}の組である。
\item[関数]\mbox{}\\
  同じ関数名で、異なった数の引数をとるような関数も定義できる。
  関数名の後ろにスラッシュと数を書くと、その数の引数をとる関数と指定できる。
  これを関数のアリティという。
\end{description}

\subsection{並行計算のための関数と構文}

Erlang並行計算の基本的な関数および構文について解説する。

\begin{description}
\item<\lstinline{spawn/1}>\mbox{}\\
  \lstinline{spawn/1}は新しいプロセスを作る関数である。新しいプロセスで行う処理が書かれた関数を受け取り、プロセスのIDを返す。\lstinline{spawn}関数はアリティが1から4までのものが定義されているが、本論文ではアリティが1のものだけを扱う。
\item<\lstinline{send/2}>\mbox{}\\
  \lstinline{send/2}は第一引数で指定したプロセスIDを持つプロセスに向けて、第二引数で指定した値を送信するという関数である。
\item<\lstinline{receive}>\mbox{}\\
  \lstinline{receive}はそのプロセスに向けて送られたメッセージを受け取り、パターンマッチを行う構文である。
\end{description}
