\chapter{背景知識}

\section{アクターモデル}

アクターモデルとは、非同期メッセージパッシングに基づいた並行計算のモデルである。
アクターと呼ばれる並行計算の主体が互いにメッセージを送り合うことで計算を進める。
対象となるアクターの名前を指定することでメッセージを送信する。


\subsection{振る舞い}

振る舞い(behavior)とは、アクターがメッセージを受け取って何をするか(action)というものである。
actionには3つのものがある。

\begin{itemize}
\item 新しいアクターを生成する
\item 他のアクターにメッセージ
\item 自分自身の振る舞いを変える
\end{itemize}


\subsection{配置}

配置(configuration)とは、システムの一瞬を切り取ったスナップショットのことである。
その時点でのアクターおよびまだ受け取られていないメッセージの集合から構成される。

\subsection{アクターの名前}

アクターのアドレスのことをアクターの名前という。
アクターの名前は一意である。


\section{定理証明支援系 Coq}

Coqとは、フランス国立情報学自動制御研究所(INRIA)で開発が進められている定理証明支援系と呼ばれるシステムの一つである。
定理証明支援系とは人間が形式的な形で証明を書きやすいように支援するシステムのことである。
定理証明支援系Coqでは以下のことが行える。

\begin{itemize}
\item プログラムの仕様の検証や数学的な命題の証明
\item 型理論に基づいた証明の機械的なチェック
\item タクティクによる柔軟な証明記述
\item プラグインシステムによる拡張
\item 他の言語へのコード抽出
\end{itemize}

\subsection{コード抽出}

CoqにはOCaml, Haskell, Schemeへのコード抽出器がデフォルトで内蔵されている。
抽出器はプラグインの形で実装されており、
