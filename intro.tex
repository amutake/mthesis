\chapter{序論}
\label{chapter:intro}

\section{研究の背景と目的}
アクターモデル\cite{}は並行計算のモデルのひとつであり、互いに非同期メッセージをやりとりするアクターと呼ばれる計算主体(computing entity)によって計算システムを表現する。
現在、Erlang\cite{}やScala\cite{}のAkka\cite{}など、アクターモデルを並行計算の基盤とした言語やライブラリが実用に供されている。
アクターモデルの形式的意味論についての研究は古くから行われているが、形式的検証は最近になっていくつかの研究成果が出てきているという状況である。
そのため、アクターモデルによって構成されたシステムの形式的な検証は喫緊の課題であると考えられる。

アクターシステムような並行システムの形式的な検証はモデル検査によるアプローチがあるが、本研究では、形式的検証のモデル検査とは別のアプローチとして定理証明支援系による検証を試みる。
定理証明支援系によるアクターモデルの形式化には、定理証明支援系Athenaによる形式化\cite{}、定理証明支援系Coqによる形式化\cite{}が試みられているが、アクターモデルによって構成されたシステムの検証には至っていない。
そこで、本研究では、定理証明支援系として定理証明支援系Coqを選択し、Coqによるアクターモデルの形式化とその形式化を用いてアクターモデルによるシステムの検証を可能にすることを目的とする。
また、実際に実行可能にするために、Erlangへの抽出も行う。


\section{本論文の構成}
本論文では、はじめに第\ref{chapter:background}章で背景知識としてアクターモデル、定理証明支援系Coq、プログラミング言語Erlangについて説明する。
第\ref{chapter:overview}章では本研究で作成した検証ライブラリActarioの概要と利用方法について説明する。
第\ref{chapter:formalization}章ではアクターモデルのCoqによる形式化手法について説明する。
第\ref{chapter:proof}章ではActarioが持つアクターシステムの性質記述および証明記述のための関数とタクティクについて説明する。
第\ref{chapter:extraction}章ではActarioが持つErlangへの抽出機構について説明する。
第\ref{chapter:related-work}章では関連研究について説明し、第\ref{chapter:conclusion}章にて今後の課題とまとめを行う。
