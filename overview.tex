\chapter{検証フレームワーク Actario}
\label{chapter:overview}

本章では、本研究で作成したアクターシステムの検証フレームワーク Actario についての概要を説明する。
Actario では、

\section{概要}

図 \ref{img:overview:workflow} は、Actario を用いてアクターシステムを検証する際のワークフローである。
まず Actario が提供するアクターシステムを記述するための記法を使って、検証したいアクターシステムを記述する。
次にそのアクターシステムが満たしているべき仕様を記述し、それを Actario が提供する証明の機構によって証明する。
最後に Actario が持つ Erlang へのコード抽出機を用いて実装を Erlang に抽出する。

\begin{figure}[tp]
  \includegraphics[width=14cm]{./img/overview/workflow.pdf}
  \label{img:overview:workflow}
  \caption{Actario のワークフロー}
\end{figure}

\section{例: 階乗計算アクターシステム}

このアクターシステムは、一つのアクターのなかで階乗を計算するのではなく、
次に何の数を掛けるかという継続を持っているアクターを生成しながら、
階乗を計算する。(?)
図 \ref{img:overview:fact} はこのアクターシステムに $3$ を与えたときの動きを表した図である。

\begin{figure}[tp]
  \includegraphics[width=14cm]{./img/overview/fact.pdf}
  \label{img:overview:fact}
  \caption{fact 3}
\end{figure}

Actario を用いると、階乗計算アクターシステムは図 \ref{code:overview:fact-impl} のように記述することができる。

\begin{figure}[tp]
  \lstinputlisting{./code/overview/fact_impl.v}
  \label{code:overview:fact-impl}
  \caption{階乗計算アクターシステム}
\end{figure}

また、証明の例として、「このアクターシステムは階乗を計算する」ということを証明する。
階乗を計算するということはつまりメッセージとして送られた任意の自然数 $n$ に対して、$n!$ を返信するということなので、
命題は図 \ref{code:overview:fact-spec} のように記述することができる。

\begin{figure}[tp]
  \begin{lstlisting}
    dummy
  \end{lstlisting}
  \label{code:overview:fact-spec}
  \caption{命題の定義}
\end{figure}

そして、この命題は図 \ref{code:overview:fact-proof} のように証明することができる。
証明の方針としては、$n$ に対しての帰納法で証明する。

\begin{figure}[tp]
  \begin{lstlisting}
    dummy
  \end{lstlisting}
  \label{code:overview:fact-proof}
  \caption{証明}
\end{figure}
