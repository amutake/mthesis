\documentclass[12pt,dvipdfmx]{jreport}
\usepackage{mthesis}
\usepackage{graphicx,xcolor,amsmath,amssymb,enumitem,url}
\usepackage{bcprules}
\usepackage{listings,lstcoq,lsterlang}

\usepackage{my}

\newtheorem{definition}{定義}
\newtheorem{theorem}{定理}
\newtheorem{lemma}{補題}

\lstdefinestyle{default}{
  language=coq,
  basicstyle={\ttfamily\normalsize},
  breaklines=true,
  % frame=tb,
  framesep=6pt,
  captionpos=b,
  numberstyle=\scriptsize,
  numbers=left,
  numbersep=6pt,
  xleftmargin=12pt,
  aboveskip=1em,
  belowskip=1em,
}
\lstdefinestyle{small}{
  style=default,
  basicstyle={\ttfamily\small},
}
\lstset{
  style=default
}

\年度{平成27年度}
\提出年月{平成28年1月}
\提出年月e{2016. ~Jan}
\題名{定理証明支援系による\\アクターシステムの検証}
\指導教員名{渡部  卓雄}
\職名{准教授}
\研究科{情報理工学}
\専攻{計算工学}
\学籍番号{14M38391}
\氏名{安武  祥平}
\内容梗概{
アクターモデルは非同期メッセージパッシング方式の並行計算のモデルのひとつであり、
アクターと呼ばれる計算主体が相互にメッセージを送り合うことで計算を進める。
アクターモデルは、アクター同士が共有する状態を持たないことから、各アクターの処理中に何らかのエラーが発生した際には
そのアクターのみを停止や再起動させることによってシステム全体は停止させずに計算を進めるということを行いやすい。
そのためアクターモデルは対障害性が必要なシステムに用いられることが多い。
しかし、そのシステムが確かにある障害に対して耐性があるということを示すのは一般に難しい。

そこで、我々はアクターシステムのための検証ライブラリActarioを作成した。
Actarioは、数学的な命題やプログラムの性質に対して厳密かつ形式的な証明を与えることができる定理証明支援系Coqのライブラリである。
Actarioは、
1. アクターモデルを使った並行アプリケーションをCoqのDSLを用いて記述できる
2. アクターモデルの意味論を提供しており、アクターモデル自体の性質を検証できる
3. 作成したアクターアプリケーションに対してその性質を検証できる
4. 作成したアクターアプリケーションをErlangのコードに変換できる
という特徴を持っている。
本論文では、簡単な例題を通してActarioの利用方法およびその実装について説明する。
}

\begin{document}
\maketitle
\tableofcontents

\chapter{序論}
\label{chapter:intro}

\section{研究の背景と目的}
アクターモデル\cite{Agha:1986aa}は並行計算のモデルのひとつであり,互いに非同期メッセージをやりとりするアクターと呼ばれる計算主体(computing entity)によって計算システムを表現する.
現在,Erlang\cite{Erlang}やScala\cite{Scala}のAkka\cite{Akka}など,アクターモデルを並行計算の基盤とした言語やライブラリが実用に供されている.
アクターモデルの形式的意味論についての研究は古くから行われているが,形式的検証は最近になっていくつかの研究成果が出てきているという状況である.
そのため,アクターモデルをサポートしている言語は耐障害性を求められるシステムに使われることが多いということもあり,アクターモデルによって構成されたシステムの形式的な検証は喫緊の課題であると考えられる.

アクターシステムのような並行システムの形式的な検証はモデル検査によるアプローチがある.
例えばアクターで記述されたシステムのモデル検査を可能にするモデル記述言語Rebeca\cite{Sirjani:2011aa}がある.
本研究では,形式的検証のモデル検査とは別のアプローチとして定理証明支援系による検証を試みる.
定理証明支援系によるアクターモデルの形式化には,定理証明支援系Athenaによる形式化\cite{Musser:2013aa},定理証明支援系Coqによる形式化\cite{Garnock-Jones:2014aa}が試みられているが,アクターモデルによって構成されたシステムの検証には至っていない.
そこで,本研究では,定理証明支援系として定理証明支援系Coqを選択し,Coqによるアクターモデルの形式化とその形式化を用いてアクターモデルによるシステムの検証を可能にすることを目的とする.
また,実際に実行可能にするために,Erlangへの抽出も行う.

アクターモデルは送り先の名前(アドレス)を指定してメッセージを送信する.したがって,アクターの名前がそのシステム内で一意になるということは重要である.
ErlangやAkkaでは,アクターの名前付けは暗黙的に行われる.つまり,プログラマが一意な名前を考える必要がない.
プログラマが明示的に名前を指定する必要があると,一意でない名前付けになる可能性があり,名前が一意でなくなるとアクターモデルの一貫性が保たれなくなる。
暗黙的に名前付けが行われると,その名前付けが必ず一意な名前を生成するならば,アクターモデルとしての一貫性は保たれる.
よって,本研究でのアクターモデルの形式化では名前付けを暗黙的に行い,そしてその名前付けの方法が必ず一意な名前付けになるということを形式的に証明する.

\section{本論文の構成}
本論文では,はじめに第\ref{chapter:background}章で背景知識としてアクターモデル,定理証明支援系Coq,プログラミング言語Erlangについて説明する.
第\ref{chapter:overview}章では本研究で作成した検証ライブラリActarioの概要と利用方法について説明する.
第\ref{chapter:formalization}章ではアクターモデルのCoqによる形式化手法について説明する.
第\ref{chapter:proof}章ではActarioが持つアクターシステムの性質記述および証明記述のための関数とタクティクについて説明する.
第\ref{chapter:extraction}章ではActarioが持つErlangへの抽出機構について説明する.
第\ref{chapter:related-work}章では関連研究について説明し,第\ref{chapter:conclusion}章にて今後の課題とまとめを行う.

\chapter{背景知識}
\label{chapter:background}

\section{アクターモデル}

アクターモデル\cite{Agha:1986aa}とは,非同期メッセージパッシングに基づいた並行計算のモデルである.
アクターと呼ばれる並行計算の主体が互いにメッセージを送り合うことで計算を進める.
対象となるアクターの名前を指定することでメッセージを送信する.


\subsection{振る舞い}

アクターモデルにおける\emph{振る舞い(behavior)}とは,アクターがメッセージを受け取って何をするか,というものである.
この「何をするか」という部分をアクターの\emph{アクション(action)}と呼ぶ.
アクターのアクションには以下の3つのものがある.
これらの組み合わせでアクターの振る舞いを表現する。

\begin{itemize}
\item 新しいアクターを生成する
\item 他のアクターにメッセージ
\item 自分自身の振る舞いを変える
\end{itemize}


\subsection{配置}

\emph{配置(configuration)}とは,アクターシステムの状態のスナップショットのことである.
その時点でのアクターおよびまだ受け取られていないメッセージの集合から構成される.
アクターモデルの操作的意味論は配置のラベル付き遷移システムとして表される。

\subsection{アクターの名前}

アクターがメッセージを送信する際に指定する宛先のことをアクターの\emph{アドレス}またはアクターの\emph{名前}という.これ以降この論文ではアクターの名前で統一する.
アクターモデルではアクターの名前を指定してメッセージを送信するため,アクターの名前が一意であるという性質はアクターモデルの一貫性のために非常に重要なものとなっている.

\subsection{\fairness}

アクターモデルでは\fairness という性質が必ず成り立つ.
\fairness とは,無限にしばしばあるアクターがあるアクションを実行できる状態にあるならば,いつかそのアクションは実行される,というものである.

\fairness が成り立たなければ,例えばあるアクターAとアクターA'が無限に相互通信を行うようなアクターシステムと,あるアクターBとアクターB'が無限に相互通信を行うようなアクターシステムを合成すると,アクターAとアクターA'のみ通信し,アクターBとアクターB'の通信は起きないということが起こりえる.
このような状況を防ぐためにも,\fairness は重要である.


\section{定理証明支援系 Coq}

Coq\cite{Coq}とは,フランス国立情報学自動制御研究所(INRIA)で開発が進められている定理証明支援系と呼ばれるシステムの一つである.
定理証明支援系とは人間が形式的な形で証明を書きやすいように支援するシステムのことである.
Coqでは以下のことが行える.

\begin{itemize}
\item プログラムの仕様の検証や数学的な命題の証明
\item 型理論に基づいた証明の機械的なチェック
\item タクティクと呼ばれる証明の方針を記述していくためのコマンドによる柔軟な証明記述
\item プラグインシステムによる拡張
\item 他の言語へのコード抽出(extraction)
\end{itemize}

例として,ある型\coqi{A}のリストを逆順にする関数\coqi{reverse}を考える.\coqi{reverse}は図\ref{code:background:reverse}のように定義できる.

\begin{figure}
\begin{lstlisting}
Fixpoint reverse (l : list A) :=
  match l with
  | [] => []
  | hd :: tl => reverse tl ++ [hd]
  end.
\end{lstlisting}
\caption{\coqi{reverse}関数の定義}\label{code:background:reverse}
\end{figure}

リストを逆順にする関数\coqi{reverse}は,二度適用するともとのリストに戻るという性質を持っているので,ここで定義した関数がその性質を満たしているということを確かめるために,証明を行う(証明できなければここで定義した関数は間違った実装になっているということになる).
Coqによるこの定理の証明は図\ref{code:background:reverse-reverse}のようになる.
ここでは補題\coqi{snoc_reverse}と目的の定理\coqi{reverse_reverse_id}を証明している.
\coqi{Lemma}および\coqi{Theorem}の後ろが命題の名前と命題を表す式になっており,
\coqi{Proof}から\coqi{Qed}までが命題を証明している部分である.証明の部分に書かれているコマンド(\lstinline{intros} や\coqi{induction}などのこと)は証明を進めていく際の方針を表すものであり,これを\emph{タクティク}と呼ぶ.

\begin{figure}
\begin{lstlisting}
Lemma snoc_reverse : forall a l, reverse (l ++ [a]) = a :: reverse l.
Proof.
  intros a l.
  induction l as [ | a' l ].
  - reflexivity.
  - rewrite IHl; reflexivity.
Qed.

Theorem reverse_reverse_id :
  forall l : list A, reverse (reverse l) = l.
Proof.
  intros l.
  induction l as [ | a l ].
  - reflexivity.
  - rewrite snoc_reverse.
    rewrite IHl.
    reflexivity.
Qed.
\end{lstlisting}
\caption{\coqi{reverse}を二度適用するともとのリストに戻ることの証明}\label{code:background:reverse-reverse}
\end{figure}


\subsection{コード抽出}

CoqにはOCaml, Haskell, Schemeへのコード抽出器がデフォルトで内蔵されている.
抽出器はプラグインの形で実装されており,Coqの型定義・関数定義からMiniMLというMLのサブセットにあたるような言語の抽象構文木に一旦変換され,そこから各言語に変換される\cite{letouzey2008extraction}.
上の\coqi{reverse}関数をOCamlに向けて変換すると,図\ref{code:background:reverse-erl}のようになる.\coqi{Extraction}コマンドでその関数を抽出することができる.
\coqi{Recursive Extraction}コマンドを使うとその関数が依存している型と関数も抽出される.

\begin{figure}
\begin{lstlisting}
Extraction reverse.
(** =>
 * let rec reverse = function
 * | Nil -> Nil
 * | Cons (hd, tl) -> app (reverse tl) (Cons (hd, Nil))
 *)
\end{lstlisting}
\caption{\coqi{reverse}の抽出}\label{code:background:reverse-erl}
\end{figure}

%% \subsection{基本的な構文}

%% \begin{figure}
%% \end{figure}
%% これを型コンストラクタと呼ぶ.型を受け取って型を返すような関数とも見れる.
%% これを値コンストラクタと呼ぶ.何か値を受け取って,値コンストラクタの型の値を返すような関数とも見れる.

\section{Erlang}

Erlang\cite{Erlang}は高可用なシステムを作るために作られた動的型付き関数型プログラミング言語である.
並行計算のモデルとしてアクターモデルを採用している.
Erlangにおいて,並行計算の主体はアクターではなくプロセスと呼ばれる.

\subsection{基本的な構文}

本論文で扱うErlangの基本的な構文について説明する.

\begin{description}
\item[変数]\mbox{}\\
  大文字から始まる識別子は変数である.Erlangでは変数は不変であり,再代入することはできない.
\item[アトム]\mbox{}\\
  小文字から始まる識別子は\emph{アトム(atom)}と呼ばれるシンボルである.
\item[組]\mbox{}\\
  波括弧で囲まれ,カンマで句切られた項は組(tuple)である.例えば\lstinline|{Var, atom, 42}|は変数\lstinline{Var},アトム\lstinline{atom},数値\lstinline{42}の組である.
\item[関数]\mbox{}\\
  関数は\lstinline[language=Erlang]{$func$($arg1$, $\cdots$, $argn$) -> $expr$.}という形で定義する.
  同じ関数名で,異なった数の引数をとるような関数も定義でき,関数を呼び出す側は関数名の後ろにスラッシュと数を書くと,その数の引数をとる関数と指定できる.
  これを関数のアリティという.
\item[パターンマッチ]\mbox{}\\
  \lstinline[language=Erlang]{case $expr$ of $pat1$ -> $expr1$; $\cdots$; $patn$ -> $exprn$ end.}という構文で式のパターンマッチを行うことができる.
\end{description}

\subsection{並行計算のための関数と構文}

Erlang並行計算の基本的な関数および構文について解説する.

\begin{description}
\item<\lstinline{spawn/1}>\mbox{}\\
  \lstinline{spawn/1}は新しいプロセスを作る関数である.新しいプロセスで行う処理が書かれた関数を受け取り,プロセスのIDを返す.\lstinline{spawn}関数はアリティが1から4までのものが定義されているが,本論文ではアリティが1のものだけを扱う.
\item<\lstinline{self/0}>\mbox{}\\
  \lstinline{self/0}は関数であり,この関数を呼び出したプロセスのIDを返す.
\item<\lstinline{!}>\mbox{}\\
  \lstinline{!}は二項演算子であり,左手で指定したプロセスIDを持つプロセスに向けて,右手で指定した値を送信するという構文である.
\item<\lstinline{receive}>\mbox{}\\
  \lstinline{receive}はそのプロセスに向けて送られたメッセージを受け取り,パターンマッチを行う構文である.
\end{description}

\chapter{検証フレームワーク Actario}
\label{chapter:overview}

本章では、本研究で作成したアクターシステムの検証フレームワーク Actario についての概要を説明する。
Actario では、

\section{概要}

図 \ref{img:overview:workflow} は、Actario を用いてアクターシステムを検証する際のワークフローである。
まず Actario が提供するアクターシステムを記述するための記法を使って、検証したいアクターシステムを記述する。
次にそのアクターシステムが満たしているべき仕様を記述し、それを Actario が提供する証明の機構によって証明する。
最後に Actario が持つ Erlang へのコード抽出機を用いて実装を Erlang に抽出する。

\begin{figure}[tp]
  \includegraphics[width=14cm]{./img/workflow.pdf}
  \label{img:overview:workflow}
  \caption{Actario のワークフロー}
\end{figure}

\section{例: 階乗計算アクターシステム}

このアクターシステムは、一つのアクターのなかで階乗を計算するのではなく、
次に何の数を掛けるかという継続を持っているアクターを生成しながら、
階乗を計算する。(?)

\begin{figure}[tp]
  \lstinputlisting{./code/overview/fact.v}
  \label{code:overview:fact}
  \caption{階乗計算アクターシステム}
\end{figure}

\chapter{アクターモデルの形式化}
\label{chapter:formalization}

本章では,Actarioにおけるアクターモデルの形式化手法について説明する.
まず,意味論の形式化,そして名前付けの方法,その名前付けの方法が一意な名前を生成するということの証明の説明を行う.


\section{意味論の形式化}

アクターモデルの操作的意味を配置のラベル付き遷移システムとして定式化する.
これ以降用いる記号を図\ref{expr:formalization:config}と定義する.
以下はその説明である.
%% また,Actarioではこれは図\ref{code:formalization:config}と定義している.

\begin{description}[style=nextline,leftmargin=12pt,parsep=0pt]
\item[\textit{Configuration}]
  アクターシステムの配置を表す.実体はアクターの集合である.
\item[\textit{Actor}]
  アクターを表す.アクターの名前,まだ実行していないアクションの列,生成番号,振る舞いのテンプレート,メッセージキューから成る.
\item[\textit{Name}]
  アクターの名前を表す.名前は,トップレベルのアクターか,生成されたアクターのどちらかであるかを表す.トップレベルのアクターとはつまり親アクターがいないアクターであり,システムの名称を表す文字列を引数にとる.また,トップレベルのアクターをトップレベルアクターと呼ぶ.生成されたアクターは,親アクターの名前と,その親アクターが何番目に生成したアクターかを表す自然数を引数にとる.
\item[\textit{Message}]
  メッセージは,アクターの名前,基本的なデータ型(自然数,文字列),およびその組み合わせ(タプル)から構成される.
\item[\textit{Behavior State}]
  アクターの振る舞いを表す.メッセージを受け取り,アクションの列を返す関数である.
\item[\textit{Actions State}]
  アクターのアクションの列を表す.アクションには,他のアクターにメッセージを送る\textsf{send},新しくアクターを生成する\textsf{new},自分自身の名前を得る\textsf{self},新しい状態になり次のメッセージの待ち状態になる\textsf{become}の4つがある.
  \textsf{become}以外の各アクションは,最後の引数にこのアクションの継続を表すものをとる.つまり,メッセージの処理は必ず\textsf{become}で終わることになる.
  また,ここでの\textit{State}は型引数であり,\textsf{become}の引数はこの型の値でなければならない.
\item[\textit{Label}]
  遷移システムのラベルを表す.\textsf{Receive},\textsf{Send},\textsf{New},\textsf{Self}はそれぞれ消費したアクションが\textsf{become},\textsf{send},\textsf{new},\textsf{self}であったときのラベルである.
\end{description}


\begin{figure}[t]
  \begin{displaymath}
    \begin{array}{rclcl}
      c & \in & \textit{Configuration} & =   & \textit{Set Actor} \\
      a & \in & \textit{Actor}  & =   & \textit{Name} \times \textit{Actions State} \times \mathbb{N} \times \\
        &     &                 &     & (\textit{State} \rightarrow \textit{Behavior State}) \times \textit{List Message} \\
      n & \in & \textit{Name}   & ::= & \textsf{toplevel}(str) \\
        &     &                 &   | & \textsf{generated}(g, n) \\
      m & \in & \textit{Message} & =  & \textit{Name} + \textit{PrimVal} + \\
        &     &                 &     & \textit{Message} \times \cdots \times \textit{Message} \\
      b & \in & \textit{Behavior State} & = & \textit{Message} \rightarrow \textit{Actions State} \\
      \alpha & \in & \textit{Actions State} & ::= & \textsf{send}(n, m, \alpha) \\
        &     &                 &   | & \textsf{new}(t, s, \kappa) \\
        &     &                 &   | & \textsf{self}(\kappa) \\
        &     &                 &   | & \textsf{become}(s) \\
      l & \in & \textit{Label}  & ::= & \textsf{Receive}(n, m) \\
        &     &                 &   | & \textsf{Send}(n, n, m) \\
        &     &                 &   | & \textsf{New}(n) \\
        &     &                 &   | & \textsf{Self}(n) \\
      \kappa & \in & \textit{Name} \rightarrow \textit{Actions State} \\
      s & \in & \textit{State} & & \\
      g & \in & \mathbb{N} & & \\
      t & \in & \textit{State} \rightarrow \textit{Behavior State} \\
      q & \in & \textit{List Message}
      str & \in & \textit{String}
    \end{array}
  \end{displaymath}
  \caption{記号の定義}\label{expr:formalization:config}
\end{figure}


\begin{figure}
  \lstinputlisting{./code/formalization/config.v}
  \caption{Actarioでの記号の定義}\label{code:formalization:config}
\end{figure}

以上の記号を用いてアクターモデルのラベル付き遷移システムは図\ref{expr:formalization:semantics}のように定式化できる.
$\overset{l}{\leadsto}$は二項関係であり,左手の配置がラベル$l$によって右手の配置に遷移するということを表している.
遷移関係は4つあり,それぞれ\textsf{Receive}ラベル,\textsf{Send}ラベル,\textsf{New}ラベル,\textsf{Self}ラベルによる遷移である.それぞれについての説明を以下に記す.

\begin{description}[style=nextline,leftmargin=12pt,parsep=0pt]
\item[$\textsf{Receive}(n_{\textrm{to}}, m)$]
  $n_{\textrm{to}}$が自身のメッセージキューの先頭にある$m$というメッセージを受け取り,そのメッセージによってアクションの列を生成する遷移である.
\item[$\textsf{Send}(n_{\textrm{from}}, n_{\textrm{to}}, m)$]
  $n_{\textrm{from}}$から$n_{\textrm{to}}$に向けてメッセージ$m$を送信するという遷移である.
  送信されたメッセージは即座に$n_{\textrm{to}}$のメッセージキューに入れられる.
\item[$\textsf{New}(n')$]
  $n'$という名前を持つアクター(ここでは子アクターと呼ぶ)を生成する遷移である.
  生成する側のアクター(ここでは親アクターと呼ぶ)の名前と生成番号によって名前がつけられる.
  親アクターは子アクターを生成した後,自身の生成番号を1増やす.
  子アクターは生成番号が0,メッセージ待ち状態でメッセージキューは空の状態で生成される.
\item[$\textsf{Self}(n)$]
  $n$というアクターが自分自身の名前を得る遷移である.
\end{description}


\begin{figure}[t]\centering
  \begin{displaymath}
    \begin{array}{rcl}
      c \cup \{(n_{\textrm{to}}, \textsf{become}(s), g, t, q)\} &
      \overset{\textsf{Receive}(n_{\textrm{to}}, m)}{\leadsto} &
      c \cup \{(n_{\textrm{to}}, t(s)(m), g, t, q)\}
      \hfill \textsc{(Receive)} \\[2ex]

      c \cup \{(n_{\textrm{from}}, \textsf{send}(n_{\textrm{to}}, m, \alpha), g, t, q),
      & &
      c \cup \{(n_{\textrm{from}}, \alpha, g, t, q), \\
      (n_{\textrm{to}}, \alpha', g', t', q')\} &
      \overset{\textsf{Send}(n_{\textrm{from}}, n_{\textrm{to}}, m)}{\leadsto} &
      \hspace{5ex} (n_{\textrm{to}}, \alpha', g', t', q' \doubleplus [m])\}
      \hfill \textsc{(Send)} \\[2ex]

      & & c \cup \{(n, \kappa(n'), g + 1, t, q), \\
      c \cup \{(n, \textsf{new}(t', s, \kappa), g, t, q)\} &
      \overset{\textsf{New}(n')}{\leadsto} &
      \hspace{5ex} (n', \textsf{become}(s), 0, t', [])\} \\
      & & \textrm{where}\ n' := \textsf{generated}(g, n)
      \hfill \textsc{(New)} \\[2ex]

      c \cup \{(n, \textsf{self}(\kappa), g, t, q)\} &
      \overset{\textsf{Self}(n)}{\leadsto} &
      c \cup \{(n, \kappa(n), g, t, q)\}
      \hfill \textsc{(Self)}
    \end{array}
  \end{displaymath}
  \caption{ラベル付き遷移システム}\label{expr:formalization:semantics}
\end{figure}

\section{名前付け}

アクターモデルではメッセージ送信の際に宛先のアクターの名前を指定してメッセージを送るため,アクターの名前が一意になるという性質は重要である.
一般的なアクターの実装では,一意な名前付けと名前の管理をするために,副作用を使った名前付けを行うことが多い.例えば,グローバル変数に自然数を持っておき,名前を生成する度にインクリメントする方法や,メモリ上に確保されるアクターのアドレスをアクターの名前にしてしまう方法がある.
しかし,Coqのような副作用を記述できない純粋な言語では,一意な名前付けは問題になりやすい.
Actarioでは,アクターの名前を,親の名前とその親が何番目に生成したアクターかを表す自然数のペアで表現することで,一意な名前付けを実現している.

本節ではこの操作的意味論の名前付けの方法によって生成された名前は必ず一意になるということを証明する.
また,これ以降,アクターが内部状態として持つ次に生成する番号を\emph{生成番号},アクターの名前に含まれる親アクターが何番目に生成したアクターかという番号を\emph{世代番号}と呼ぶ.

\subsection{名前の一意性の証明}

生成された名前の一意性の証明に入る.
これを証明するために,\emph{\transinv(trans invariant)}という遷移の間で変わらない名前に関する性質を定義する.
\transinv は配置に関する述語であり,以下のように3つの述語、\emph{\chain (chain)}, \emph{\fresh (gen fresh)}, \emph{\nodup (no\_dup)}の組で定義される.

\begin{definition}{\transinv}
\begin{displaymath}
  \begin{array}{l}
    \texttt{trans\_invariant}(c) :=
    \texttt{chain}(c) \wedge \texttt{gen\_fresh}(c) \wedge \texttt{no\_dup}(c)
  \end{array}
\end{displaymath}
\end{definition}

\texttt{chain}, \texttt{gen\_fresh}, \texttt{no\_dup}の簡単な説明は以下のとおりである.

\begin{description}[style=nextline,leftmargin=12pt,parsep=0pt]
\item[\chain]
  配置内の各アクターについて,そのアクターが他のアクターから生成されたものであるなら親アクターはその配置に存在している.
\item[\fresh]
  配置内の各アクターについて,そのアクターが次に生成するアクターの名前は配置内で新しい名前である.
\item[\nodup]
  配置内のすべてのアクターの名前は一意である.
\end{description}

証明の説明に入る前に,この節で用いる関数をいくつか定義する.


\begin{description}[style=nextline,leftmargin=12pt,parsep=0pt]
\item[\texttt{parent} $: \textit{Actor} \rightarrow \textit{Actor}$]
  アクターを受け取り,そのアクターの親アクターを返す.親がいない場合は,\texttt{nothing} を返す.
\item[\texttt{gen\_number} $: \textit{Actor} \rightarrow \mathbb{N}$]
  アクターを受け取り,そのアクターの名前の世代番号を返す.
  トップレベルアクターの場合は,\texttt{nothing} を返す.
\item[\texttt{next\_number} $: \textit{Actor} \rightarrow \mathbb{N}$]
  アクターを受け取り,そのアクターの生成番号を返す.
\item[\texttt{name} $: \textit{Actor} \rightarrow \textit{Name}$]
  アクターを受け取り,そのアクターの名前を返す.
\item[\texttt{names} $: \textit{Set(Actor)} \rightarrow \textit{Set(Name)}$]
  アクターの集合を受け取り,その集合の各アクターの名前の集合を返す.
\end{description}

\subsubsection{\chain}

まず,\transinv の一つ目の述語である\emph{\chain (chain)}という,配置に関する述語を定義する.
\chain とは,配置の中に含まれるアクターについて,そのアクターが他のアクターによって生成されたものであるならば,そのアクターも配置の中に含まれる,という述語である.
つまり,トップレベルアクターを起点とする木になるということである.
ここで,トップレベルアクターが複数ある場合でも,トップレベルアクター以外で親アクターが存在しないアクターが存在しなければ\chain は成り立つということに注意されたい.
\chain は以下のように定義される.

\begin{definition}{\chain}
\begin{displaymath}
  \begin{array}{l}
    \texttt{chain}(c) :=
    \forall a \in c, \forall p, p = \texttt{parent}(a) \Rightarrow p \in c
  \end{array}
\end{displaymath}
\end{definition}

ここで,任意の遷移について\chain は保存される\chainpreserv を証明する.
証明は遷移のラベルによる場合分けで行う.
\chain はアクターの名前によってのみで決定され,かつ名前が変更されうる遷移ラベルは\textsc{New}しかないので,\textsc{New}のみ考えればよい.

\begin{lemma}{\chainpreserv}
\begin{displaymath}
  \begin{array}{l}
    \forall c, c' \in \textit{Configuration}, \forall l \in \textit{Label},
    \texttt{chain}(c) \wedge c \overset{l}{\leadsto} c' \Rightarrow \texttt{chain}(c')
  \end{array}
\end{displaymath}
\end{lemma}

%% \begin{figure}[tp]
%%   \lstinputlisting{./code/formalization/chain_preservation.v}
%%   \label{code:formalization:chain-preservation}
%%   \caption{連鎖保存性の証明}
%% \end{figure}

\subsubsection{\fresh}

次に,\fresh を定義する.
\fresh は任意のアクターが次に生成する名前は必ず一意になるという述語であるが,
Actarioではこれを以下のように定義することで実現している.
%% We define \texttt{gen\_fresh} predicate that, for each actor in the configuration, the name of its child is always fresh.
%% The definition of \texttt{gen\_fresh} is complicated a little.
%% We translate the proposition that next generated name is fresh to the following.

\begin{definition}{\fresh}
\begin{displaymath}
  \begin{array}{l}
    \texttt{gen\_fresh}(c) := \\
    \quad \forall a \in \texttt{actors}(c), \forall p \in \texttt{actors}(c), p = \texttt{parent}(a) \Rightarrow \\
    \quad \quad \quad \texttt{gen\_number}(a) < \texttt{next\_number}(p)
  \end{array}
\end{displaymath}
\end{definition}

アクターの名前は生成するアクターの名前とその生成番号のペアで生成されるため,この\fresh が成り立っていれば必ず一意な名前を生成するということは,あるアクターの名前とそのアクターの生成番号からなる名前を持つようなアクターが存在しないことから成り立つ.
ただし,次に生成するアクターの名前は一意でもその次に生成するアクターは一意ではないこともある.
例えば,同じ名前でかつ次の生成番号も同じという2つのアクターがいた場合,まず片方のアクターが生成するアクターの名前は一意だが,その次にもう片方のアクターがアクターを生成したとすると,名前が被ってしまう(これを第一パターンとする).
また,親アクターがシステム内に存在せずに,親の親は存在しているという場合,親の親が次に生成するアクターの名前は被らないが,その子アクターが次に生成する名前は他のアクターの名前と被ってしまう可能性がある(これを第二パターンとする).
このように,遷移前で\fresh が成り立っていても,いくつか遷移したあとで一意ではない名前を生成することはありうる.
以上から,遷移後も\fresh が成り立つようにするためには,遷移前は\fresh という条件だけでは足りない.
第一パターンは\nodup を,第二パターンは\chain を前提に含めることで防ぐことができるので,
\freshpreserv は以下のような補題となる.

\begin{lemma}{\freshpreserv}
\begin{displaymath}
  \begin{array}{l}
    \forall c, c' \in \textit{Configuration}, \forall l \in \textit{Label}, \\
    \quad \texttt{chain}(c) \wedge \texttt{gen\_fresh}(c) \wedge \texttt{no\_dup}(c) \wedge c \overset{l}{\leadsto} c' \Rightarrow \\
    \quad \texttt{gen\_fresh}(c')
  \end{array}
\end{displaymath}
\end{lemma}

\subsubsection{\nodup}

最後に\nodup を定義する.
一意性は,与えられた配置のすべてのアクターの名前が一意であるという述語である.
これは最終的に証明したい名前の一意性の定義と同じである.
一意性は以下のように定義できる.

\begin{definition}{\nodup}
\begin{displaymath}
  \begin{array}{l}
    \texttt{no\_dup}(c) :=
    \forall a \in c, \texttt{name}(a) \notin
    \texttt{names}(c \setminus \{a\})
  \end{array}
\end{displaymath}
\end{definition}

\chainpreserv,\freshpreserv と同様に,\noduppreserv も定義する.
遷移後も\nodup が成り立つためには遷移前の配置が\nodup と\fresh を満たす必要がある.

\begin{lemma}{\noduppreserv}
\begin{displaymath}
  \begin{array}{l}
    \forall c, c' \in \textit{Configuration}, \forall l \in \textit{Label}, \\
    \quad \texttt{gen\_fresh}(c) \wedge \texttt{no\_dup}(c) \wedge c \overset{l}{\leadsto} c' \Rightarrow \texttt{no\_dup}(c')
  \end{array}
\end{displaymath}
\end{lemma}

\subsubsection{名前の一意性の証明}

ここまでで\transinv の定義ができたので,この意味論によって動的に生成される名前が一意になるということの証明を行う.
まず,遷移の間で\transinv が保存されるという\emph{\transinvpreserv (trans invariant preservation)}を証明する.
\transinv は\chain,\fresh,\nodup から成るので,
\chainpreserv,\freshpreserv,\noduppreserv から,\transinvpreserv は明らかに成り立つ.

\begin{lemma}{\transinvpreserv}
  \begin{displaymath}
    \begin{array}{l}
      \forall c, c' \in \textit{Configuration}, \forall l \in \textit{Label}, \\
      \quad \texttt{trans\_invariant}(c) \wedge c \overset{l}{\leadsto} c' \Rightarrow
      \texttt{trans\_invariant}(c')
    \end{array}
  \end{displaymath}
\end{lemma}

次に初期状態について\transinv が成り立っていれば,任意回の遷移後も\transinv が成り立つという補題($\transinvpreserv^{\star}$)を証明する.

\begin{lemma}{$\transinvpreserv^{\star}$}
  \begin{displaymath}
    \begin{array}{l}
      \forall c, c' \in \textit{Configuration}, \forall l \in \textit{Label}, \\
      \quad \texttt{trans\_invariant}(c) \wedge c \overset{l}{\leadsto\star} c' \Rightarrow
      \texttt{trans\_invariant}(c')
    \end{array}
  \end{displaymath}
\end{lemma}

ここで,$c \overset{l}{\leadsto\star} c'$ は遷移の反射推移閉包である.
この証明は遷移についての帰納法と\transinvpreserv によって証明できる.

最後に,初期状態について\transinv が成り立っていれば名前の一意性は成り立つということを証明する.
\transinv は\nodup から成るので,$\transinvpreserv^{\star}$から明らかである.
\begin{theorem}{名前の一意性}
  \begin{displaymath}
    \begin{array}{l}
      \forall c, c' \in \textit{Configuration}, \forall l \in \textit{Label}, \\
      \quad \texttt{trans\_invariant}(c) \wedge c \overset{l}{\leadsto\star} c' \Rightarrow \texttt{no\_dup}(c')
    \end{array}
  \end{displaymath}
\end{theorem}

\chapter{証明機構}
\label{chapter:proof}

本章では、Actarioでのアクターシステムに関する命題の定義方法およびその証明の機構について説明する。
まず遷移パスを使って行う\fairness の形式化について説明する。
次に\fairness の形式化を行う際に用いた、この先いつかある事柄が成り立つ、というような述語を使って、初期状態からいつか必ずあるラベルで遷移するというような性質を表す述語を定義する。
その後、証明を行う際に用いる遷移可能なラベルと遷移後の配置を計算するための関数について説明し、
最後に遷移パスをつかって証明を行う方法について説明する。

\section{\fairness の形式化}
\fairness とは、アクターモデルが持つ性質で、無限にしばしば遷移しうるラベルは必ずいつかそのラベルで遷移するというものである。
アクターモデルでは必ず\fairness が成り立っており、これを前提としなければ証明できない性質もあるため、まず\fairness の形式化を行う。
また、\fairness を形式化する際に使う、この先いつかある事柄が成り立つ、というような述語はActarioの証明機構でも使っているので、その説明も行う。

通常、\fairness を表現する際には時相論理が必要になるが、Coqは時相論理はサポートしていない。
そのため、配置の遷移列である遷移パスを使って\fairness を表現する。
この手法はAppl$\pi$\cite{}で用いられている手法である。Appl$\pi$は$\pi$計算のためのライブラリであるが、\fairness の定義方法についてはアクターモデルに対しても同様に用いることができる。

\subsection{遷移パス}
遷移パスは自然数$\mathbb{N}$から\texttt{option config}型への関数として定義する (図~\ref{code:formalization:path})。
定義域の自然数は、初期状態から何回目の遷移によってこの配置になったかという番号である。この番号をインデックスと呼ぶ。
値域はそのインデックスに対応する配置を表す。\texttt{config}型ではなく\texttt{option config}型になっているのは、これ以上遷移ができないパスも表したいからである。つまり、これ以上遷移ができない配置のインデックスを$n$とすると、$\forall m > 0, n + m$に遷移パス関数を適用した結果は\texttt{None}になる。

\begin{figure}[tp]
\begin{lstlisting}
Definition path := nat -> option config.
\end{lstlisting}
  \label{code:formalization:path}
  \caption{遷移パスの定義}
\end{figure}

また、遷移パスの仕様を、パスに対する制約の形で定義する(図\ref{code:formalization:path-spec})。
すべてのインデックス$i$について、$i$番目の配置が存在するならば、$i+1$番目の配置が存在するならそれは遷移できるものか、それ以上遷移できない。$i$番目の配置が存在しないならば、その次の配置も存在しない、という意味である。

\begin{figure}[tp]
\begin{lstlisting}
Definition is_transition_path (p : path) : Prop :=
  forall i,
    (forall c, p i = Some c ->
       ((forall c' l, ~ (c ~(l)~> c')) -> p (S i) = None) \/
       (exists c', p (S i) = Some c' -> exists l, c ~(l)~> c')) /\
    (p i = None -> p (S i) = None).
\end{lstlisting}
  \label{code:formalization:path-spec}
  \caption{遷移パスの仕様}
\end{figure}

配置はアクターの集合であるため、本来はその中でアクターの順序はない。しかし、Actarioでの定義はアクターのリストになっており、順序がある。
ラベル付き遷移システムの各遷移の定義(付録\ref{appendix:trans})では入れ替えて等しいものとしているため、遷移パスの各配置も順序を入れ替えられるようにしてよい。
この命題は以下のようになるが、遷移パスは実際には関数であるためこの命題は証明できない(関数の返り値は引数に応じて一意に決まるため)。
よってこれは公理とする(図\ref{code:proof:path-perm})。

\begin{displaymath}
  \begin{array}{l}
    \forall c, c' \in \textit{Configuration}, permutation(c, c') \rightarrow \\
    \quad \forall p \in \textit{Path}, \forall i \in \mathbb{N}, \\
    \quad \quad \quad p(i) = c \rightarrow p(i) = c'
  \end{array}
\end{displaymath}

\begin{figure}[tp]
\begin{lstlisting}
Axiom path_perm :
  forall p i c c',
    is_transition_path p ->
    Permutation c c' ->
    p i = Some c ->
    p i = Some c'.
\end{lstlisting}
  \label{code:proof:path-perm}
  \caption{遷移パスの配置の入れ替え}
\end{figure}

\subsection{\enabled}
次に、与えられた配置が与えられたラベルでもって遷移ができるという述語を定義する。これを\emph{\enabled (enabled)}と呼ぶ。
Actarioでは、\enabled はある配置からあるラベルによって遷移した先の配置が存在すると定義する (図\ref{code:formalization:enabled})。

\begin{figure}[tp]
\begin{lstlisting}
Definition enabled (c : config) (l : label) : Prop :=
  exists c', c ~(l)~> c'.
\end{lstlisting}
  \label{code:formalization:enabled}
  \caption{\enabled}
\end{figure}

\subsection{Inifinitely Often Enabled}
次に、無限にしばしば\enabled になるという述語infinitely often enabledを定義する(図\ref{code:proof:infinitely-often-enabled})。
これは、すべてのインデックス$i$について、$i$番目の配置があるラベルによって遷移可能であるならば、その先そのラベルによって遷移が可能になる配置が存在する、という意味である。

\begin{figure}
\begin{lstlisting}
Definition infinitely_often_enabled (l : label)
                                    (p : path) : Prop :=
  forall i c, p i = Some c ->
    enabled c l ->
    exists i' c',
      i < i' /\
      p i' = Some c' /\
      enabled c' l.
\end{lstlisting}
\label{code:proof:infinitely-often-enabled}
\caption{infinitely often enabled}
\end{figure}

\subsection{Eventually Processed}
次に、いつかあるラベルによって遷移が起きるという述語eventually processedを定義する(図\ref{code:proof:eventually-processed})。
これはこの遷移パス内でラベル$l$による遷移が存在するという意味である。

\begin{figure}
\begin{lstlisting}
Definition eventually_processed (l : label) (p : path) : Prop :=
  exists n c c',
    p n = Some c /\ p (S n) = Some c' /\ c ~(l)~> c'.
\end{lstlisting}
\label{code:proof:eventually-processed}
\caption{eventually processed}
\end{figure}


\subsection{Fairness}
以上を使って、\fairness は図\ref{code:formalization:fairness}のように定義できる。
任意の遷移パスと任意のラベルにおいて、もし無限にしばしばそのラベルで遷移可能になるならば、いつかはそのラベルでの遷移が起きるという意味である。
なお、\coqi{is_postfix_of}という述語は、あるパスがあるパスのはじめの$n$回の遷移を除いたものであり、\fairness の定義においてはこれから先常にこれが成り立つことを表すために使われている。
もし\coqi{is_postfix_of}がなければ、そのパス全体ではあるラベルで遷移が起こることは保証されるが、遷移が起こったあとは無限にしばしば遷移可能になっても遷移しないようなパスも許される。
これを防ぐために、\coqi{is_postfix_of}を用いて、実際に遷移が起きてもその先無限にしばしば遷移可能になることがあればその部分でもいつかは遷移が起きるということを強制している。

\begin{figure}
\begin{lstlisting}
Definition is_postfix_of (p' p : path) : Prop :=
  exists n, (forall m, p' m = p (m + n)).

Definition fairness : Prop :=
  forall p p', is_postfix_of p' p ->
    (forall l,
      infinitely_often_enabled l p' ->
      eventually_processed l p').
\end{lstlisting}
\label{code:formalization:fairness}
\caption{\fairness の定義}
\end{figure}

\section{命題の定義}

本節では、アクターモデルによるシステムの性質を記述するための述語について説明する。
アクターシステムの性質を表す際に便利な述語を、前節で定義した\coqi{eventually_processed}を使って定義する。
アクターモデルによって構成されたシステムは、何回遷移が起きるかわからないがいつかはこの状態になってほしい、もしくはこのアクションが実行される、というような性質を検証したいことが多い。
よって図\ref{code:proof:ev-do-label}のように、初期状態とラベルの述語とする。

\begin{figure}
\begin{lstlisting}
Definition eventually_do_label (c0 : config) (l : label) :=
  forall p : path,
    p 0 = Some c0 ->
    is_transition_path p ->
    eventually_processed l p.
\end{lstlisting}
\label{code:proof:ev-do-label}
\caption{性質に関する述語の定義}
\end{figure}

\section{遷移可能なラベルと遷移後の配置の計算}

アクターシステムの性質の証明は、初期状態から考えられる遷移を一つずつ追っていくという方針にする。
しかし、素朴に一つずつの遷移を追うと、ユーザーが遷移毎に遷移前の配置から遷移後の配置を予想して書き出す必要が出てしまう。
こうすると証明が非常に煩雑になってしまうため、配置から遷移可能なラベルの集合、配置から各ラベルによって遷移した後の配置は決定可能であることに着目し、まず遷移前の配置から遷移可能なラベルの集合を計算し、その各ラベルで遷移した後の配置を計算することで、ユーザが配置の内容を書き出さずとも、遷移パスを追いやすくする。
このように計算可能な形にすることは、Microsoft Researchによって開発されており四色定理の証明などに使われているCoqの拡張ライブラリであるSsreflect\cite{ssreflect}の考え方と同一である。

\subsection{遷移可能なラベルの計算}

まず遷移可能なラベルの集合の計算について説明する。
配置はアクターの集合として表されており、また、\ref{chapter:formalization}で説明したように、各アクターの名前は必ず一意となる。
ラベルはアクター型の\coqi{remaining_actions}フィールドにある\coqi{actions}型の値に一対一になっているため、各アクターはアクションを行って遷移を行うか、遷移できないかのどちらかになる。
よって、配置からラベルの集合への関数を作ることができる。
この関数を\coqi{possible_labels}と呼ぶことにする。
\coqi{possible_labels}の定義は図\ref{code:proof:possible-labels}のようになる。\coqi{cat_options}はヘルパー関数で、\coqi{option A}型のリストを、\coqi{Some}の場合だけ抜き出したようなリストに変換する関数である。
\coqi{possible_labels}の計算結果のラベルのリストが遷移可能なラベルとして網羅されているということはまだ証明できておらず、今後の課題となっている。


\begin{figure}
\begin{lstlisting}
Fixpoint cat_options {A : Type} (opts : seq (option A)) :=
  match opts with
  | [::] => [::]
  | None :: opts' => cat_options opts'
  | Some a :: opts' => a :: cat_options opts'
  end.

Definition possible_labels (c : config) : seq label :=
  cat_options (map (fun a =>
    match a with
    | {| actor_name := to; remaining_actions := become _; queue := msg :: msgs |} =>
      Some (Receive to msg)
    | {| actor_name := fr; remaining_actions := send to msg _ |} =>
      if to \in (map actor_name c) then Some (Send fr to msg) else None
    | {| actor_name := p; remaining_actions := new _ temp ini _; next_num := nx |} =>
      Some (New (generated nx p))
    | {| actor_name := me; remaining_actions := self _ |} =>
      Some (Self me)
    | _ => None
   end) c).
\end{lstlisting}
\label{code:proof:possible-labels}
\caption{\coqi{possible_labels}の定義}
\end{figure}

\subsection{遷移後の配置の計算}

あるラベルによって遷移した後の配置の計算について説明する。
ラベルとそのラベルによって遷移した後の配置は集合として同型となるものを除いて一意となるため、集合として同型なもののなかの一つを選択するようにすれば、これも関数として定義できる。
この関数を\coqi{calc_trans}と呼ぶ。
関数定義は少々煩雑なため付録\ref{appendix:calc-trans}にある。
\coqi{calc_trans}の計算結果の配置が確かにもとの配置からそのラベルによって遷移したものになっているということはまだ証明できておらず、今後の課題となっている。

\section{証明の方針}

\subsection{trace path}

\lstinline{possible_labels}、\lstinline{calc_trans}を使って遷移パスを追うための補題を用意する。
これは、ある遷移パスについて、$i + 1$番目の配置は$i$番目の配置から遷移可能であるもののいずれかである、という意味である。
$i$番目の配置はその配置から計算されたラベルのリストのどれかで遷移するはずであるので、結論部分はそのすべてを選言によって結合したものになっている。
もしラベルのリストが空ならば、次の配置はない。
この際、集合としては同型でもリストとしては異なるものは除かれており、厳密に遷移後のパスが網羅されているわけではないが、図\ref{code:proof:path-perm}の\coqi{path_perm}によってこの問題は考えなくてよくなっている。
この補題を使うと、遷移パスの制約の追加をCoqの計算に任せることができる。
また、この補題もまだ証明されておらず、今後の課題となっている。


\begin{figure}
\begin{lstlisting}
Fixpoint any1 {A : Type} (p : A -> Prop) (d : Prop) (s : seq A) :=
  match s with
  | [::] => d
  | [:: h] => p h
  | h :: t => p h \/ any1 p d t
  end.

Lemma trace_path :
  forall p i c,
    is_transition_path p ->
    p i = Some c ->
    any1 (fun l => p (S i) = Some (calc_trans c l)) (* exhaustive by path_perm *)
         (p (S i) = None)                      (* if possible_labels is empty *)
         (possible_labels c).
\end{lstlisting}
\label{code:proof:trace-path}
\caption{\coqi{trace_path}の定義}
\end{figure}

\subsection{タクティク}

Actarioが提供する、遷移パスを使った証明に便利なタクティクについて説明する。
図\ref{code:proof:tactics}はタクティクの定義である。
まず\coqi{step}というタクティクは、あるパスが遷移パスになっているという前提と、そのパスの$i$番目の配置が$c$であるという前提から、$i + 1$番目としてありうる配置を計算し前提に加えるというものである。これは\coqi{trace_path}を使っている。
\coqi{step_until_stop}は、これ以上遷移できなくなるまで遷移を追っていき、その際に得たパスに対する制約をすべて前提に加えるというものである。遷移が無限に続く場合はタクティクの実行は終わらない。
\coqi{found}は、パスの$i$番目から$i + 1$番目に遷移する際に、証明したいラベルによって遷移が起きたことがわかったときに使い、証明を終わらせることができる。

\begin{figure}
\begin{lstlisting}
Ltac step p_is_path p :=
  move/(_ _ _ _ p_is_path p): trace_path;
  rewrite/calc_trans/=.

Ltac step_until_stop is_path p0 :=
  let P := fresh "p" in
  try (progress step is_path p0=> P; step_until_stop is_path P).

Ltac found i p p' :=
  exists i; eexists; eexists;
    split; last split; [ apply p | apply p' | ];
    (eapply trans_receive || eapply trans_send ||
      eapply trans_new || eapply trans_self);
    apply/Permutation_refl.
\end{lstlisting}
\label{code:proof:tactics}
\caption{タクティク}
\end{figure}

\subsection{証明の方法}

非決定的な遷移をせずかつ遷移が無限に続かないような場合は\coqi{step_until_stop}タクティクを実行し\coqi{found}タクティクによって証明を終わらせることができる。
非決定的な遷移があるか遷移が無限に続くような場合は\coqi{step}によって遷移を追っていく。

\chapter{Erlang へのコード抽出}

本章では、Actario が持っている Erlang へのコード抽出機構について説明する。

\section{マッピング}

\subsection{型のマッピング}

代数的データ型はタプルで表現する。

例えば、\texttt{nat} 型は Erlang では以下の表記になる。


\begin{lstlisting}
  O -> {o}
  S O -> {s, {o}}
  S (S O) -> {s, {s, {o}}}
\end{lstlisting}

\subsection{アクションのマッピング}

アクションの部分のマッピングのみ特別扱いし、対応する Erlang のコードにマッピングする。

\subsection{相互再帰関数}

Erlang ではトップレベルの

\chapter{関連研究}
\label{chapter:related-work}

\section{\applpi}

\applpi\cite{Affeldt200817}は、Affeldtらによって構築された、Coqによる$\pi$計算の検証フレームワークである。
Actarioは\applpi を非常に参考にしており、Actarioにおけるいくつかのアイデア、例えばライブラリの使い方(ライブラリが提供する構文でアプリケーションを記述し、その性質をライブラリが提供する関数や型を使って書き証明、その後実行可能なプログラミング言語のコードに抽出すること)やアクションを継続渡し形式にすることなどは、\applpi で用いられていたものである。

\section{Verdi}

Verdi\cite{Verdi}は、耐障害性を持つ分散システムのための検証フレームワークである。
Verfiの特徴は、Verdiを利用して記述した分散システムとその性質について、障害がないネットワークを仮定して証明することができれば、その性質を保ったまま、あるパケットの消失や重複、ノードの突然停止などの障害を含むようなネットワークに耐性のあるシステムを生成するというものである。
かなり実用的な例についてもVerdiを用いて検証されており、合意形成アルゴリズムであるRaftの検証や、それを使った分散Key-Value Storeの検証なども行っている。
Actarioはまだそのような実用的な例での検証には至れていないが、将来の目標としてこのようなシステムの検証も行いたい。
現段階におけるActarioのメリットとしてはよりアクターモデルをサポートしている言語に近い形でアプリケーションを記述できること、Erlangに変換できること、の二点が挙げられる。


\section{Athenaによる形式化}

定理証明支援系によるアクターモデルの形式化としてMusserとVarelaによる定理証明支援系Athena\cite{Athena}による形式化\cite{Musser:2013aa}が挙げられる。
これはアクターモデル自身の性質の検証とアクターモデルの理解を助けることを目的とした形式化である。
Actarioとの主な違いは、ユーザーが一意な名前をつけなければならない点である。
Actarioでは、第\ref{chapter:formalization}章で説明したように、ユーザーが決めずにシステムが生成し、そして名前が一意になるということは形式的に証明されている。

\section{Network Calculusの形式化}

Network Calculus\cite{Garnock-Jones:2014aa}とはアクターモデルを拡張したモデルである。Network Calculusの形式化と性質の証明にCoqが用いられている。
参考文献\cite{Garnock-Jones:2014aa}ではNetwork Calculusの形式化の前段階として純粋なアクターモデルの形式化を行っている。
このアクターモデルの形式化も遷移システムとして形式化しているが、配置の遷移は非決定的ではなく決定的になっている。
一般的に、並行システムは実行順が非決定的になるので、Actarioでは非決定的遷移としている。

\chapter{今後の課題と結論}

\section{今後の課題}

\section{まとめ}


\chapter*{謝辞}
\addchapter{謝辞}
本研究を行うにあたり、ご指導いただきました渡部卓雄准教授に深く感謝致します。
また、様々な助言をくださった渡部研究室の皆様に感謝致します。
%% 渡部研究室の同期に感謝致します。を入れたいが、

\bibliographystyle{jplain}
\bibliography{./references}

\end{document}
