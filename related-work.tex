\chapter{関連研究}
\label{chapter:related-work}

\section{\applpi}

\applpi\cite{Affeldt200817}は、Affeldtらによって構築された、Coqによる$\pi$計算の検証フレームワークである。
Actarioは\applpi を非常に参考にしており、Actarioにおけるいくつかのアイデア、例えばライブラリの使い方(ライブラリが提供する構文でアプリケーションを記述し、その性質をライブラリが提供する関数や型を使って書き証明、その後実行可能なプログラミング言語のコードに抽出すること)やアクションを継続渡し形式にすることなどは、\applpi で用いられていたものである。

\section{Verdi}

Verdi\cite{Verdi}は、耐障害性を持つ分散システムのための検証フレームワークである。
Verfiの特徴は、Verdiを利用して記述した分散システムとその性質について、障害がないネットワークを仮定して証明することができれば、その性質を保ったまま、あるパケットの消失や重複、ノードの突然停止などの障害を含むようなネットワークに耐性のあるシステムを生成するというものである。
かなり実用的な例についてもVerdiを用いて検証されており、合意形成アルゴリズムであるRaftの検証や、それを使った分散Key-Value Storeの検証なども行っている。
Actarioはまだそのような実用的な例での検証には至れていないが、将来の目標としてこのようなシステムの検証も行いたい。
現段階におけるActarioのメリットとしてはよりアクターモデルをサポートしている言語に近い形でアプリケーションを記述できること、Erlangに変換できること、の二点が挙げられる。


\section{Athenaによる形式化}

定理証明支援系によるアクターモデルの形式化としてMusserとVarelaによる定理証明支援系Athena\cite{Athena}による形式化\cite{Musser:2013aa}が挙げられる。
これはアクターモデル自身の性質の検証とアクターモデルの理解を助けることを目的とした形式化である。
Actarioとの主な違いは、ユーザーが一意な名前をつけなければならない点である。
Actarioでは、第\ref{chapter:formalization}章で説明したように、ユーザーが決めずにシステムが生成し、そして名前が一意になるということは形式的に証明されている。

\section{Network Calculusの形式化}

Network Calculus\cite{Garnock-Jones:2014aa}とはアクターモデルを拡張したモデルである。Network Calculusの形式化と性質の証明にCoqが用いられている。
参考文献\cite{Garnock-Jones:2014aa}ではNetwork Calculusの形式化の前段階として純粋なアクターモデルの形式化を行っている。
このアクターモデルの形式化も遷移システムとして形式化しているが、配置の遷移は非決定的ではなく決定的になっている。
一般的に、並行システムは実行順が非決定的になるので、Actarioでは非決定的遷移としている。
